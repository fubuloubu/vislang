\section{Introduction}

VisLang is a block diagram language designed to allow fast and
easy prototyping of programs for embedded processors. The language is created
with a graphical editor in mind, and the core language is designed to be extensible
so that any graphical editor can add additional elements or attributes for graphical
display or other features.

\subsection{Key Language Features}

The language itself is based on the idea of blocks: small parts that can be grouped
together into ever larger blocks and re-used across different programs. VisLang has a
small group of fundamental (or atomic) blocks that will be understood by the VisLang
compiler. Other blocks will be constructed as groupings of these atomic blocks, and can
be referenced in other files. Libraries of useful function blocks can be constructed
from these atomic blocks containing common parts such as timers, latches, etc. The 
ability to include blocks from libraries and other programs is a standard feature of
the language.

The syntax of VisLang leverages standard XML, giving the language a well-formed and 
machine readable backbone. As noted previously, the point of leveraging XML is so that
3rd party programs can manipulate the file format in an easy way, and so that external
programs can add additional elements (e.g. visual information for display) and attributes
(e.g. location information) to the existing set of elements and attributes defined by
the language. Those additional tags not included in the list of recognized elements/
attributes will be ignored by the compiler as a valid program is only defined as a series
of well-connected blocks. All parts have a set of necessary attributes, and all connections
require the source to exist. This creates a natural flow to interpreting the language, such
that only functional errors should be raised by the compiler during compilation.

The VisLang Compiler will parse and check the specified input file and generate a viable
C source file that can be used in combination with other generated and manually created
C source files to combine into fully functional programs for embedded devices. Each
generated file contains code that is completely reusable as the generated code has standard
interfaces and does not rely on global definitions. Some manual coding will still be
necessary to link into different types of embedded devices, the point is to create good
intermediate code such that linking to I/O devices and dealing with the nuances of an
arbitrary embedded device can be minimized.
