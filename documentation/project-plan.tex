\section{Project Plan}

Since I was working on this project alone, there was more autonomy in creating the
language. This actually led to be a bit of a problem as my initial ideas for what
I wanted to accomplish were unrealistic and I was more willing to slide on the schedule
I set for myself since there were no other group members to act in the project manager
role to keep things on schedule, nor were any group members available to ensure that
project goals were reasonable. Regardless, after an initial development period of over
a month a working front end was developed leveraging the XML specification available
online with the planned tags and attributes I had at the time. It was eventually decided
that adding my own namespace for XML tags would be necessary to reduce the processing
load in the scanner and parser section to work with other elements. This is right around
the time the XML Abstract Syntax Tree was fully developed and work started on the backend
of the compiler.
\par
At this point, there was little testing in existance since I was just attempting to parse
the example program, so testing had to be approached. It was decided I was going to
leverage to bash testing script from the MicroC example language provided in class, and
have additional python scripts be created leveraging the Ctypes module to test the
functionality of each test case. Once this was decided, the first MWE test case was
developed (the buffer test case) and more work was done to get that to pass. More
complicated test cases led to a decision to add a complete block parsing algorithm in
order to be able to produce a correctly formatted program for code generation. After
some development, this algorithm allowed more test cases to pass and work to continue
on integrating block group and referencing functionality. Once this was completed, the
initial draft of VisLang was considered feature complete, and several other planned
features were descoped due to time constraints on the project.

\subsection{Software Development Environment}

Development for the project took place entirely on an Asus Chromebook C720 using crouton
to enable a full linux environment. The tools used for this project are listed below:

\begin{itemize}
\item ubuntu 14.04.2 LTS (Operating system environment)
\item git 1.9.1 (source code, test, and documentation configuration management)
\item vim 7.4.52 (general purpose text editor)
\item ocaml 4.01.0 (including ocamlyacc and ocamllex)
\item gcc 4.8.2 (compiling generated code)
\item python 2.7.6 (scripting language for testing compiled C objects)
\end{itemize}

\subsection{Project Timeline}

\begin{itemize}
\item 2015-05-27  Decided on Simulink-like block language, using XML syntax
\item 2015-05-31  Created example program
\item 2015-06-12  Proposal Submitted
\item 2015-06-21  First draft of scanner
\item 2015-06-26  First draft of parser
\item 2015-06-30  Scanner working for all attributes and tags
\item 2015-07-04  LRM Submitted
\item 2015-07-08  Parser working for new ast
\item 2015-07-09  Added top level
\item 2015-07-10  Moved errors to their own module
\item 2015-07-14  XML ast working
\item 2015-07-17  Integrated blockification function
\item 2015-07-24  Removed interpreter
\item 2015-07-27  Simplified blockification process
\item 2015-07-28  Moved trace algorithm from blockify to it's own module
\item 2015-08-05  Working Code Generation for all atomic parts
\item 2015-08-12  Got blocks completely working end-to-end
\item 2015-08-12  Updated blockification for Reference part
\item 2015-08-13  Began working on paper
\item 2015-08-14  Submitted paper
\item 2015-08-15  Celebrated from Canada
\end{itemize}

\subsection{Project Log}
\inputlog{../git.log}
