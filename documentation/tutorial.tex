% TODO
\section{Language Tutorial}

The first group of elements (INPUT, OUTPUT, CONSTANT, NAME) provide named
memory locations for use in the application. All of these elements have a
type attribute where the datatype of the name is specified. INPUT, OUTPUT,
and SIGNAL have a scope attribute, which defines whether the name is local
scope (block-wide), global scope (program wide), or device scope (only
Inputs and Outputs can be device scope). When device scope is chosen, an
IO address must be chosen to specify where to read or write the I/O data
to the interface with the target hardware. The CONSTANT element describes
a named constant, which is initialized to the value provided and will reside
in memory and read each pass during program execution. All of these
elements have a size attribute, which means they can refer to array quantities
as well. A structure can be used as a datatype.

The second group of elements (PROGRAM, BLOCK, CONNECTION) describes block
constructs. This is the basis of how the program works. The PROGRAM element
is a container for the entirety of the program. The PROGRAM element has a
name attribute which describes what the compiled program should be
called by the compiler. The BLOCK element is also a container, but used
throughout the program to encapsulate functionality or define helper parts
in referenced files. BLOCK has an optional reference property which describes
the location of a block in a file using the "/path/to/file.vs$\vert$path$\vert$to$\vert$block"
syntax, where the vertical bars describe the heirarchy required to discover
the referenced block. Any referenced block will have to have connections to 
all of the defined inputs in the referenced block. The outputs are not
required.

The CONNECTION element is how blocks and parts are connected
together. The "to" attribute describes where the connection is headed, and
cannot have a different CONNECTION element with a matching "to" attribute
name. The "from" attribute describes where the connection came from, and
does not have the same restriction. Connections can either be made to named
signals such as INPUTS, OUTPUTS, CONSTANTS, or SIGNAL, or they can be made
implicitly to blocks themselves using "block$\vert$input" or "block$\vert$output" syntax.

The third group is the atomic parts. The Language Reference Manual will go
into more detail here but all of these parts should form the bare minimum 
required by the base language to compose all other required functions from.
All of the atomic parts operate on specific types and have a specific defined
function present in the compiler. All of the blocks have one or more inputs
(except DT) and have exactly 1 output. All of the blocks are time invariant,
,e.g. given the same input they produce the same output, except the DT
and MEM blocks. The MEM block will store the value of something until the next
pass. The DT block outputs the delta time between each pass. Since all blocks
are simple and operate on a defined amount of inputs and outputs, they can be
used with arrays to define parallel operations, e.g. applying the OR gate
operation against two boolean arrays of the same size to produce a third boolean
array of that same size. Structures will not work as inputs to these parts.

All the logic gates besides the NOT gate (AND, OR, etc.) as well as the BITWISE,
SUMMER, and PROD elements are defined as having two or more inputs (BITWISE can
be one if a mask is set). The way this works in practice is that each successive
input increments the number after the word "input" e.g. "input1", "input2",
"input3", etc. when making connections to these parts.

The fourth group describes array and structure utilities. MUX creates arrays from
a group of inputs with the same type, and similarly DEMUX will deconstruct an
array into a group of outputs of the same type. Similar to the atomic gate parts,
these parts allow on the relevant inputs/outputs a variable amount of elements
to be specified using the input\# syntax on the connection elements. The size of
arrays will be static during compilation, and array size will be a quantity traced
between blocks to verify size matching between usages.

CONSTRUCT and DESCTRUCT create and decompose STRUCT elements respectively. The
way this works is that a STRUCT element is defined seperately, with member
attributes defining the name and type of each member of the struct. The CONSTRUCT
element will allow a set of inputs matching the type of each of the members of a
struct to create a new named quantity of that structure, and similarly DESTRUCT
deconstructs a structure into signals that match each of those members. The STRUCT
element can be defined or referenced through a file, allowing a seperate definition
file to be reused over a project. Additionally, a STRUCT can be created from or written
to an array of certain types, using the exact amount of memory required with buffer
space as necessary to represent it as a chunk of memory. This will allow message
definitions to be encoded or decoded to/from a hardware buffer. Lastly, an array
can be composed of struct elements, as long as all elements match that structure
definition.

The last group of elements is the array functions. These functions operate on
arrays per element of the array, and require a referenced function of a certain
number and type of inputs/outputs to translate the input array into the output.
MAP will apply the referenced function of a single input to a single output
(that may or may not matches the input's type). What is important is that the input
array's type and the type of the input to the function match, as well as the
same criteria for the output.

FILTER is similar to MAP in that it has a referenced function, however the output is
required to be a boolean conditional. What FILTER will do is use that conditional
output to filter the input array into an output array. Since there is a design
decision not to allow variable array sizes, FILTER requires a size attribute for
the output, and will not update the output if the size does not match what the
attribute requires. This can be useful for parsing a buffer of word-based definitions,
as we can filter the array of words to find a single UID that is a part of the word
definition and pass that as the output to a parsing function.

The REDUCE element is similar to MAP and FILTER in that it takes an array
argument and use the function reference, however this function will take each
element of the array and apply it to a function that takes two inputs: the first
being the same type as each array element, and the second matching the output type
of the function. REDUCE will start with the first element and the initial value
for the output and recursively run through each element of the array, feeding back
the output as the inital value input for the next iteration. When REDUCE has
exhausted all of the elements of the array, the final output of the function
is returned back as the output of the REDUCE element.

Most attributes for elements are defined with a string. In some cases, it may be 
preferable to use a literal definition for a value instead of creating a reference
or using a named quantity. Every datatype in VisLang has a way of creating a literal.
Examples of literals for all VisLang types are below:
\begin{longtable}[c]{ |c|c| } 
    \hline
    datatype & example literal(s) \\ 
    \hline\hline
    boolean & false, true \\
    single, double & 1.000, 1e-6, 100 \\
    integer & 100, 0x20, 2x1011, 8x671, -120 \\ 
    \hline
    address & 0x1A56, 0x10A8$\vert$14 \\
    arrays & $[1, 2, 3, 4]$ \\
    structures & \{1, \{2, 3\}\} \\
    \hline
\end{longtable}

Booleans can only be false or true. Floating point quantities (single, double) can
be a decimal quantity (with or without significant digits after the decimal point),
or specified using scientific notation e.g. 10e6. Floating point literals can be
specified with a negative sign as well e.g. -1.234e-6. Integer quantities (e.g.
uint8, uint16, uint32, uint64, int8, int16, int32, int64) can be specified as a
decimal quantity (cannot have a decimal point), or using hex (e.g. 0x1A), binary
(e.g.2x1010), or octal (e.g. 8x2462) representation. Only signed integers can have
a negative sign in front. Additionally, for address literals, specifying a bit of
and address can be done with the $\vert$ operator against a hex representation of
the address (e.g. 0x1235$\vert$7). Address literals can only be in hex. Array
literals are created with the bracket operators (e.g. $[true, true, false]$), and
structure literals can be created using the braces operator (e.g. \{0x10, \{1, false\}\}.

Any attribute that accepts a literal can also take a reference to a named value,
in a similar way that block references are defined (e.g. "/path/to/file.vs$\vert$path$\vert$to$\vert$value").

\subsection{Example Program}

The following program illustrates some of the major features of the language. The
program itself takes a Digital Input on the target device, reads it, and starts a
timer when the input is enabled. When the timer counts up to the target time, it
will set a Digital Output true and reset the timer, creating a fast-blinking light
with a period of 2 seconds.

\srccode{xml}{../example/timed-blinking-light/timed-blinking-light.vl}
    {example:top}{Example Top Level}

As noted, the above file contained a reference to another part called timer defined
in timer.vs in the same directory. Any references must take place on a relative path
to that file, and that reference must contain the same number of inputs specified by
the target file inside the file referencing that part. The number of outputs need
not match, but any outputs specified in the file referencing that part must also
match what is available from the target file or an exception will be thrown. All
other unused outputs will be disregarded. The following file displays the target
file, complete with the relevant inputs and outputs as specified/required by the
previous file.

\srccode{xml}{../example/timed-blinking-light/timer.vl}
    {exmaple:timer}{Example Referenced Block}
