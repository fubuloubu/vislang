\section{VisLang Compiler Architecture}

\begin{wrapfigure}[25]{r}[-1.2cm]{0.25\textwidth}
  \centering
  \begin{tikzpicture}[auto]
    \node [block]                   (source)  {Source File (XML)};
    \node [block,below of=source]   (xmltok)  {XML Tokens};
    \node [block,below of=xmltok]   (xmlobj)  {XML Object Tree};
    \node [block,below of=xmlobj]   (blkobj)  {Block Object Tree};
    \node [block,below of=blkobj]   (optblk)  {Optimized Block Tree};
    \node [block,below of=optblk]   (gencod)  {Auto-generated Code};

    \draw[->] (source) -- node{xscanner.mll}    (xmltok);
    \draw[->] (xmltok) -- node{xparser.mly}     (xmlobj);
    \draw[->] (xmlobj) -- node{blockify.ml}     (blkobj);
    \draw[->] (blkobj) -- node{blockparse.ml}   (optblk);
    \draw[->] (optblk) -- node{compile.ml}      (gencod);
  \end{tikzpicture}
  \caption{VLCC Architecture}
  \label{arch:vlcc}
\end{wrapfigure}

The architecture of VisLang has two distinct stages of operation from source file to
target file. The scanning and parsing stages of the front end essentially implement
read the XML elements of interest and skip through parsing any unrecognized tokens.
After a correctly formed XML Object Tree has been formed, the next step is to translate
that tree of XML Objects (an XML Object has a tag, a list of attributes and a list of
inner objects, if any) into a block tree where each block can verify and access the
necessary attributes it should have. Each object can also see the list of connections
assigned to it when it was parsed, which is important when verifying the program is
well-formed. That block tree is then taken and re-organized such that the inner objects
of a block are in Static Single Assignment form, e.g. each block can be computed using
the outputs of previous blocks in the list for that containing block.
\par
In the process of reorganizing the inner blocks, the Block Parse algorithm will also
perform the check that the inner blocks align (e.g. they call blocks that are properly
assigned, they match in datatype, etc.) and that only inner blocks which are used to
compute the output are in the calculation. Any blocks which do not align will raise an
error (datatype mismatch, incorrectly attributes for that object, etc.), any blocks
which reference other blocks in a circular fashion will raise an error (algebraic loop),
and any blocks that are not necessary to compute an output will be optimized away. The
end result is that the remaining optimized block tree is a suitable candidate to be
directly translated into generated code as that generated code will have the property
of minimal side-effects: all computations are computed either from inputs or derived
from inputs.
