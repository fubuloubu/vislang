\section{Language Tutorial}

Any well-formed VisLang program can be constructed with the following guidance. Firstly,
the outer-most set of tags (the top of the XML tree or Top Block) must be a BLOCK element
and should be given a name appropiate to the intended functionality (file name and block
name do not have to match). Next, a selection of INPUT and OUTPUT elements should be chosen
for that block that describe how it will interface with other blocks or C source files.
When those have been chosen and given names and datatypes, it is now time to design the
inner logic of the block being made.
\par
Any collection of parts can be strung together
from any input to any output. The first important caveat to this is that the datatype
of each input and output must match to the corresponding connection being made. All parts
in VisLang have an explicit or implicit datatype, and that must be matched up as suggested
to avoid compilation errors. The second important caveat is that all calculations avoid
referencing themselves. This means that when tracing from any output to any input, there
is no instance of a calculation being used to define itself unless a MEM block has been
placed to prevent an algebraic loop occurance.
\par
Next comes the decision if there will be any references to external blocks.
The user can specify the location of an external block in a file using the following syntax:
"/path/to/file.vl$\vert$path$\vert$to$\vert$block". When using external blocks, it is
important to match the names and datatypes of the inputs/outputs to that block when
making connections to that block. Any incorrect types or names will be flagged as an error
at compilation time. Any BLOCK in any VisLang file can be referenced, but each file
must be compiled separately to avoid runtime errors when attempting to compile the target
generated C files.
\par
Finally, the design of the program could have grown to sufficient complexity where
encapsulating that functionality into a separate block would be desireable. At that point,
encapsulating all of the chosen parts into another BLOCK part would allow that part to
be isolated from other parts in the program (different namespace), and that part can be
referenced into another block in that program or any other.
\par
There are a few specific things about some of the parts VisLang provides worth noting.
All of the logic gate elements besides the NOT gate (AND, OR, etc.) and the SUMMER and
PROD elements are defined as having two or more inputs. The way this works in practice
is that each successive input increments the number after the word "input" e.g. "input1",
"input2", "input3", etc. when making connections to these parts. If a number is skipped
or the count does not start at 1, a compilation error will be raised. These parts are
known as "binary recursive" parts because the operation involved will be applied to every
input to the block in a recursive fashion e.g. (input1 op (input2 op (...))).
\par
Most attributes for elements are defined with a string. The name attribute is common to
every part in the VisLang language. The connection elements can reference these names
when making connections between any two blocks. If an element is an atomic part (any
element besides BLOCK and REFERENCE), then linking a block input to that name is as easy
as using that block's name. This is due to the fact that all of the atomic parts in VisLang
are defined as only having one output, so there is no ambiguity. The other attributes have
more explicit values that must match what is specified in the LRM.
\par
When using the "ic" or "value" attributes, values they require are literals of the relevant
datatype. Examples of literals for all VisLang types are below:
\begin{longtable}[c]{ |c|c| } 
    \hline
    datatype & example literal(s) \\ 
    \hline\hline
    boolean & false, true \\
    single, double & 1.000, -100., .000 \\
    integer & 100, 0x20, 2x1011, 8x671, -120 \\ 
    \hline
\end{longtable}

Note booleans can only be false or true. Floating point quantities (single, double) can
be a decimal quantity (with or without significant digits after the decimal point),
or specified using a decimal point e.g. 10.600. Floating point literals can also be
specified with a negative sign as well e.g. -1234.00. Integer quantities (e.g.
uint8, uint16, uint32, uint64, int8, int16, int32, int64) can be specified as a
decimal quantity (cannot have a decimal point), or using hex (e.g. 0x1A), binary
(e.g.2x1010), or octal (e.g. 8x2462) representation. Only signed integer datatypes can
have a negative sign in front.

\subsection{Example Program}

The following program illustrates some of the major features of the language. The
program itself takes an Input (presumably on the target device) and starts a
timer when the input is enabled. When the timer counts up to the target time, it
will set the Output true and reset the timer, creating a pulse-blinking light
with a period of 2 seconds.

\srccode{xml}{../example/timed-blinking-light.vl}{example:top}{Example Top Level}

As noted, the above file contained a reference to another part called timer defined
in timer.vs in the same directory. Any references must take place on a relative path
to that file, and that reference must contain the same number of inputs specified by
the target file inside the file referencing that part. The number of outputs need
not match, but any outputs specified in the file referencing that part must also
match what is available from the target file or an exception will be thrown. All
other unused outputs will be disregarded. The following file displays the target
file, complete with the relevant inputs and outputs as specified/required by the
previous file.

\srccode{xml}{../example/timer.vl}{example:timer}{Example Referenced Block}
