\section{Conclusion}

The VisLang compiler was moderately a success because it lays the groundwork for future
iterations of the program for use in a fully optimized environment as a replacement for
developing embedded programs using proprietary IDEs or programming languages that are
more difficult to understand. A variety of lessons where learned during the development
of the program that will be detailed below. As a result of several of the lessons learned,
suggestions for future development are also presented.

\subsection{Lessons Learned}

The original idea for VisLang was very ambitious: to make a general purpose embedded
computing language in a visual format that could be used to develop programs for
small embedded devices such as the Aruduino platform. Early on the in the project
it was realized that this is much to ambitious of a goal because that would mean 
essentially replicating the avr C libraries in a language that was not meant for it.
Instead, the scope of VisLang was first pared down such that VisLang instead generated
C code instead of assembly so that it would be easy to link with the already feature-
complete libraries that exist for the platform. Integration with the avr libraries
remains untested at this time, but it is easy to show how VisLang-generated code could
be easily integrated into while loop almost any embedded device utilizes to run code.
The thought is that the specialty code needed to interface with the device is only a
small portion of the overall code the user is interested in running, so such a tradeoff
would be acceptable.
\par
Another lesson that was learned a few weeks into development of VisLang was that testcase
driven development would be required to move forward at an acceptable pace. Originally,
the development philosophy was trying to implement all of the required features of the
language at once, but even trying to get the simplest program (a buffer block, which passes
input directly to output) was a difficult task and a philosophy change was needed. A test
case for the buffer block was written and the compiler was made to work appropiately with
that test case first before further test cases were developed and more functionality was
implemented to meet those new cases. The benefits of this approach primarily are that these
test cases are available for quick turnover later on to validate future changes to the code.
This happened several times where a change made to satisfy one test case ultimately did,
but broke several of the already completed test cases. Integrating test cases into
development was one of the biggest lessons learned at first.
\par
Finally, the last lesson that was learned was to complete more preliminary work before
creating a specification for a language. Knowing how much would be possible as well as
prototyping some of the features beforehand would have helped to write a much more sound
specification to begin with so that scope reduction and philosophy changes would be
minimized.

\subsection{Future Improvements}

Several pieces of VisLang's original specification were descoped for the initial version
of the compiler due to time constraints. Given future development time, most of these
features would be required for VisLang to reach it's full potential as a general purpose
prototyping and embedded controller language to match potential rivals such as Simulink
and Modelica.
\par
First off, a graphical interface for manipulating VisLang code would make development of
programs in the language much easier, since that was the original intended use-case.
Significant development would be necessary here, but thankfully true to the original
design goals development to VisLang and any GUI environment that might use the language
could happen mostly in parallel. Specific attention would need to be taken to overhaul
VisLang's front end to make it truly resilient to non-VisLang XML elements. As it stands
right now, VisLang supports ignoring additional attributes, but using attributes of the
same name can confuse the parser which will throw an error. Either an alternative way to
specify VisLang attributes would need to be attempted, the VL Compiler would need to be
hardened against those attributes by more clever design of the front end, or a better
methodology of specifying the XML would need to be investigated to satisfy this goal.
\par
Arrays and Structures would be essential to truly allowing the language to prosper in
all of its intended use cases. Originally, the Array features of VisLang would allow a
user to create and pass around Arrays to inputs, enabling Function Language elements such
as Filter, Reduce, and Map to be applied so duplicate functionality can be performed with
minimal coding. This is important to larger embedded devices because they typically have
redundant interfaces that require the exact same processing to each element. Additionally,
digital busses can be arrays of packet structures that need the exact same processing
where a language that operated on them in parallel would be able be more efficient in its
operation. Structures would be used in a similar way, enabling I/O messages to be stripped
apart and processed in a predictable way, or output messages to be created in a specified
manner.
\par
Of course, a block language like VisLang can always support more parts. The original
specification for VisLang included several parts that were deemed unnecessary for the
initial implementation of the compiler, so identifying and adding that functionality
would be an obvious next step for the language. Adding the ability to encapsulate or link
to arbitrary code would also be another possible design goal for VisLang as often it is
necessary to have a calculation drive some action that interfaces with the embedded
processor, such as servicing the watchdog timer or managing interrupts. This would be
important if VisLang were to be used on larger projects.
