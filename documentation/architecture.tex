% TODO
\section{VisLang Compiler Architecture}

%The architecture for vislang is as follows:
%xml syntax -> xml scanning -> xml parsing -> xml syntax tree 
%  -> blockification -> block object tree 
%  -> runtime -> bytecode 
%  -> compilation
%
%the xml syntax tree is the literal representation of the incoming xml syntax
%the block object tree is the literal translation of the block objects
%bytecode is the translation of the BOT through the runtime into a stack-based representation of the program
%compilation produces a valid c program from bytecode

\begin{figure}[!htb]
\begin{tikzpicture}[auto]
    \node [block]                   (source)  {Source File (XML)};
    \node [block,below of=source]   (xmltok)  {XML Tokens};
    \node [block,below of=xmltok]   (xmlobj)  {XML Object Tree};
    \node [block,below of=xmlobj]   (blkobj)  {Block Object Tree};
    \node [block,below of=blkobj]   (optblk)  {Optimized Block Tree};
    \node [block,below of=optblk]   (gencod)  {Auto-generated Code};

    \draw[->] (source) -- node{xscanner.mll}    (xmltok);
    \draw[->] (xmltok) -- node{xparser.mly}     (xmlobj);
    \draw[->] (xmlobj) -- node{blockify.ml}     (blkobj);
    \draw[->] (blkobj) -- node{blockparse.ml}   (optblk);
    \draw[->] (optblk) -- node{compile.ml}      (gencod);
\end{tikzpicture}
\caption{VLCC Architecture}
\label{arch:vlcc}
\end{figure}
