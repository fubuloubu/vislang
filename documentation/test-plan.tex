\section{Test Plan}

The generated target code for \autoref{example:top} and \autoref{example:timer}
from the Tutorial are below:

\srccode{C}{../example/timed-blinking-light.c}{gen:top}{Generated Code for Top Level}
\srccode{C}{../example/timer.c}{gen:timer}{Generated Code for Referenced Block}

The output program, when compiled using gcc, will be able to process the inputs provided
by 'connecting' to that block and update it's outputs over time for every iteration of
the program in the main loop. For programs without MEM or DT blocks, the resulting code
has the property of being time-invariant, that is no matter how many times it is called
or whatever the duration between calls are, it will produce the exact same result every
time. The MEM element will remember a value between the last call and the current such
that the resulting program loses that time invariance, but this operation allows the
production of functionality such as states and transfer functions to be modeled using
VisLang. The DT element is used when the amount of time between calls is important,
but for a steady system this should never be an issue as it should stay relatively
constant. This means programs using DT may or may not be almost time invariant, but
that depends on the usage of the block.
\par
To automate testing of VisLang programs, a shell script (\autoref{test:script}) was
borrowed from the MicroC example language. The shell script looks at all of the files
in a directory and processes them into 1 of 3 testing groups: test, pass, fail. The pass
and fail testing groups simply looks to verify that the source file for such a test case
either pass compilation (pass cases) or fails compilation (fail). In this way, specific
compiler features that have to do with processing the input file (instead of the code
generated) can be checked without further complication. The 'test' cases first verify that
the source file can generate the target file, but additionally a functional check is
provided through associated *.in and *.out files that are run against the target file.
\par
The methodology for testing these cases involves additionally compiling the target files
as source files for gcc, and turning the resulting object file into a shared library that
can be interpreted through a testing script. The testing script is a python script that
is generated using the -d option of vlcc which takes the *.in file and runs a while loop
over each line of the file and produces what the output of the program would be for each
timestep. The timestep is purposely never updated to ensure that a repeatable test
environment exists. The output produced by the test script is then compared against the
associated *.out file to see if any differences exist. If the two files match, then the
test is determined to be passing.
\par
The following is a list of the test cases used to verify the VisLang compiler produces
correct code:

\begin{longtable}[c]{ |r|p{6cm}| }
    \hline
    Test Case                   & Description \\
    \hline
    \hline
    \listingref{test:failal}    & Shows that an algebraic loop is caught \\
    \hline
    \listingref{test:failbc}    & Shows that a badly specified connection is caught \\
    \hline
    \listingref{test:failma}    & Shows that a missing attributes in a block is caught\\
    \hline
    \listingref{test:failub}    & Shows that an unended block is caught \\
    \hline
    \listingref{test:passeb}    & Shows that an empty block compiles okay \\
    \hline
    \listingref{test:failceb}   & Shows that multiple empty blocks inside each other are okay\\
    \hline
    \listingref{test:passg}     & Shows that random XML and other input is okay between tags \\
    \hline
    \listingref{test:testb}     & Shows proper operation of buffer block (O = I)\\
    \hline
    \listingref{test:testbib}   & Shows that a block within a block works \\
    \hline
    \listingref{test:testm}     & Shows the memory block works okay \\
    \hline
    \listingref{test:testc}     & Shows all the comparision operations work \\
    \hline
    \listingref{test:testg}     & Shows all the logical gates work \\
    \hline
    \listingref{test:testmc}    & Shows all math blocks work okay \\
    \hline
    \listingref{test:tesths}    & Shows a reference block works okay \\
    \hline
    \listingref{test:testsrl}   & Shows that a complex block (SR Latch) works \\
    \hline
    \listingref{test:testt}     & Shows that example (Timer block) works \\
    \hline
    \caption{Test Case Descriptions}
    \label{table:testcases}
\end{longtable}
